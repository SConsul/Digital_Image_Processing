\documentclass{article}
\usepackage[utf8]{inputenc}
\usepackage{bm}
\usepackage{relsize}
\usepackage{amsmath,amssymb,amsthm, mathrsfs}
\usepackage{geometry}
\title{CS 663 Assignment 5}
\author{Sarthak Consul - 16D100012\\
Parthasarathi Khirwadkar - 16D070001\\
Bhishma Dedhia - 16D170005}
\date{}
\geometry{
    a4paper,
    total={170mm,257mm},
    left=12mm, 
    right=12mm, 
    top=20mm,
    }
\begin{document}

\maketitle

\centerline{\textbf{\large\large{{{QUESTION 1}}}}}
\begin{align*}
    g_1 = f_1 + h_2*f_2 &\longleftrightarrow G_1 = F_1 + H_2F_2\\
    g_2 = f_2 + h_1*f_1 &\longleftrightarrow G_2 = F_2 + H_1F_1\\
    G_1 &= F_1 + H_2(G_2 - H_1F_1)\\
    G_1 &= (1-H_1H_2)F_1 + H_2G_2\\
    F_1 &= \frac{G_1 - H_2G_2}{1 - H_1H_2}\\
    F_2 &= \frac{G_2 - H_1G_1}{1 - H_1H_2}
\end{align*}
This formula will be a problem when \textbf{H1H2 = 1}. In other words whenever the product of the respective blurring kernels are near unity, then the estimated images \textbf{F1} and \textbf{F2} will blow up for the particular frequency in the Fourier domain. This will give us extremely inaccurate images on taking the inverse Fourier transform.

As this instability (ie. \textbf{H1H2 = 1}) can occur at arbitrary frequencies, there is no simple way (eg. a low pass filter like in Wiener filters) of negating the effects of these unstable points. 

\end{document}