\documentclass{article}
\usepackage[utf8]{inputenc}
\usepackage{bm}
\usepackage{relsize}
\usepackage{amsmath,amssymb,amsthm}
\usepackage{geometry}
\renewcommand\thesection{\Alph{section}}
\title{CS 663 Assignment 4}
\author{Sarthak Consul - 16D100012\\
Parthasarathi Khirwadkar - 16D070001\\
Bhishma Dedhia - 16D170005}
\date{}
\geometry{
    a4paper,
    total={170mm,257mm},
    left=12mm, 
    right=12mm, 
    top=20mm,
    }
\begin{document}

\maketitle

\centerline{\textbf{\large\large{{{QUESTION 6}}}}}


\section{}

\begin{align*}
% \centering
    \bm{\bm{y}}^\intercal\bm{P}\bm{y} &= \bm{y}^\intercal (\bm{A}^\intercal\bm{A})\bm{y} \\
    &= (\bm{A}\bm{y})^\intercal(\bm{A}\bm{y})\\
    &= \lvert\lvert \bm{A}\bm{y} \rvert\rvert_2^2 \geq 0 \hspace{5pt}(\because \lvert\lvert\bm{v} \rvert\rvert_2^2 \geq 0 \hspace{2pt}\forall \bm{v} \in \mathbb{R}^m)
\end{align*}

Let $\bm{u}$ be an eigenvector of $\bm{P}$ with corresponding eigenvalue of $\lambda$,
\begin{align*}
    &\hspace{5pt}\bm{P}\bm{u} = \lambda \bm{u}\\
    &\implies \bm{u}^\intercal(\bm{P}\bm{u}) = \lambda \bm{u}^\intercal\bm{u}\\
    &\implies \lambda =\frac{\bm{u}^\intercal\bm{P}\bm{u}}{\lvert\lvert\bm{u} \rvert\rvert_2^2} \geq 0 \hspace{5pt}(\because \lvert\lvert\bm{v} \rvert\rvert_2^2 \geq 0 \hspace{2pt}\forall \bm{v} \in \mathbb{R}^n)
\end{align*}

\begin{align*}
    \bm{z}^\intercal\bm{Q}\bm{z} &= \bm{z}^\intercal (\bm{A}\bm{A}^\intercal)\bm{z} \\
    &= (\bm{A}^\intercal\bm{z})^\intercal(\bm{A}^\intercal\bm{z})\\
    &= \lvert\lvert \bm{A}^\intercal\bm{z} \rvert\rvert_2^2 \geq 0\hspace{5pt}(\because \lvert\lvert\bm{v} \rvert\rvert_2^2 \geq 0 \hspace{2pt}\forall \bm{v} \in \mathbb{R}^n)
\end{align*}

Let $\bm{v}$ be an eigenvector of $\bm{Q}$ with corresponding eigenvalue of $\mu$,
\begin{align*}
    &\hspace{5pt}\bm{Q}\bm{v} = \mu \bm{v}\\
    &\implies \bm{v}^\intercal(\bm{Q}\bm{v}) = \mu \bm{v}^\intercal\bm{v}\\
    &\implies \mu =\frac{\bm{v}^\intercal\bm{Q}\bm{v}}{\lvert\lvert\bm{v} \rvert\rvert_2^2} \geq 0 \hspace{5pt}(\because \lvert\lvert\bm{v} \rvert\rvert_2^2 \geq 0 \hspace{2pt}\forall \bm{v} \in \mathbb{R}^m)
\end{align*}
Both $\bm{P}$ and $\bm{Q}$ are said to be positive semi-definite matrices. \\

\raggedright
\section{}
% \centering
\begin{align*}
    \bm{Q}(\bm{Au}) &= (\bm{A} \bm{A}^\intercal)(\bm{A}\bm{u}) \hspace{5pt}(\because \bm{A}\bm{A}^\intercal = \bm{Q})\\
    &= \bm{A} (\bm{A}^\intercal \bm{A})\bm{u}\\
    &= \bm{A}(\bm{P}\bm{u}) \hspace{5pt}(\because \bm{A}^\intercal \bm{A} = \bm{P})\\
    &= \lambda (\bm{A}\bm{u}) \hspace{5pt}(\because \bm{P}\bm{u} = \lambda \bm{u})
\end{align*}
$\bm{u}$ is an eigenvector to a nxn matrix and so has n elements.
\begin{align*}
    \bm{P}(\bm{A}^\intercal\bm{v}) &= (\bm{A}^\intercal\bm{A}) (\bm{A}^\intercal\bm{v}) \hspace{5pt}(\because \bm{A}^\intercal\bm{A} = \bm{P})\\
    &= \bm{A}^\intercal (\bm{A}\bm{A}^\intercal)\bm{v}\\
    &= \bm{A}^\intercal(\bm{Q}\bm{v}) \hspace{5pt}(\because \bm{A}\bm{A}^\intercal = \bm{Q})\\
    &= \mu (\bm{A}^\intercal\bm{v}) \hspace{5pt}(\because \bm{Q}\bm{v} = \mu \bm{v})
\end{align*}
$\bm{v}$ is an eigenvector to a mxm matrix and so has m elements.\\

\raggedright
\section{}
% \centering
Let the eigenvalue of $\bm{Q}$ corresponding to $\bm{v_i}$ by $\mu_i$
\begin{align*}
    \bm{A}\bm{u_i} &= \bm{A}\frac{\bm{A}^\intercal \bm{v_i}}{\lvert\lvert\bm{A}^\intercal \bm{v_i} \rvert\rvert_2}\\
     &= \frac{\bm{A}\bm{A}^\intercal \bm{v_i}}{\lvert\lvert\bm{A}^\intercal \bm{v_i} \rvert\rvert_2}\\
     &= \frac{\bm{Q} \bm{v_i}}{\lvert\lvert\bm{A}^\intercal \bm{v_i} \rvert\rvert_2}\\
    &= \frac{\mu_i \bm{v_i}}{\lvert\lvert\bm{A}^\intercal \bm{v_i} \rvert\rvert_2}\\
     &= \frac{\mu_i}{\lvert\lvert\bm{A}^\intercal \bm{v_i} \rvert\rvert_2} \bm{v_i}\\
\end{align*}
As $\bm{Q}$ is positive semi-definite, $\mu_i \geq 0$ (equality results in $\bm{u_i} = \bm{0}$), also the L2-norm of any non-zero vector is always positive.\\
So \begin{equation*}
    \gamma_i = \frac{\mu_i}{\lvert\lvert\bm{A}^\intercal \bm{v_i} \rvert\rvert_2}
    \begin{cases}
    \geq 0, & \text{if }\bm{A}^\intercal \bm{v_i} \neq \bm{0}\\
    = 0, & \text{if} \bm{A}^\intercal \bm{v_i} = \bm{0}
  \end{cases}
\end{equation*}

So, there exists $\gamma_i \geq 0$ such that $\bm{Au_i} = \gamma_i \bm{v_i}$\\

\raggedright
\section{}
% \centering
From the result obtained in part C,
$\bm{A}\bm{u}_i = \gamma_i \bm{v}_i $, $\gamma_i \geq 0 \hspace{10pt}\forall i=1,2,\dots,m$
\begin{align*}
    \bm{U}\bm{\Gamma} &= 
        \begin{bmatrix}
        \bm{v}_1| & \bm{v}_2|& \bm{v}_3|& \dots |& \bm{v}_m
        \end{bmatrix}
        \begin{bmatrix}
    \gamma_1 & 0 & 0 & \dots & 0\\
    0 & \gamma_2 & 0 & \dots & 0\\
    0 & 0 & \gamma_3 & \dots  & 0\\
    \vdots & \vdots & \vdots & \ddots & \vdots\\
    0 & 0 & 0 & 0 & \gamma_m
\end{bmatrix}\\\\
&= \begin{bmatrix}
        \gamma_1\bm{v}_1| & \gamma_2\bm{v}_2|& \gamma_3\bm{v}_3|& \dots |& \gamma_m\bm{v}_m
        \end{bmatrix}\\
&= \begin{bmatrix}
        \bm{Au}_1| & \bm{Au}_2|& \bm{Au}_3|& \dots |& \bm{Au}_m
        \end{bmatrix}\\
&= \bm{A}\begin{bmatrix}
        \bm{u}_1| & \bm{u}_2|& \bm{u}_3|& \dots |& \bm{u}_m
        \end{bmatrix}\\  
&=\bm{AV}
\end{align*}
\[\bm{u}_i^\intercal\bm{u}_j =\frac{\bm{v}_i^\intercal\bm{A}\bm{A}^\intercal \bm{v_j}}{\lvert\lvert\bm{A}^\intercal \bm{v_i} \rvert\rvert_2 \lvert\lvert\bm{A}^\intercal \bm{v_j} \rvert\rvert_2} = 
\begin{cases}
0, & \text{for } i\neq j\\
1, & \text{for } i=j
\end{cases}
\]
\vspace{5pt}
Thus $\bm{V}\bm{V}^\intercal = \bm{I_n}$ (i.e.$\bm{V}$ is orthonormal)
\begin{align*}
    \therefore \bm{A}\bm{V}\bm{V}^\intercal &= \bm{U\Gamma}\bm{V}^\intercal\\
    \bm{A}&= \bm{U\Gamma}\bm{V}^\intercal
\end{align*}
% \begin{align*}
%     \bm{U\Gamma V^\intercal} &= \bm{U}
% \begin{bmatrix}
%     \gamma_1 & 0 & 0 & \dots & 0 \\
%     0 & \gamma_2 & \dots & 0 & 0 \\
%     0 & 0 & \gamma_3 & \dots  & 0 \\
%     \vdots & \vdots & \vdots & \ddots & \vdots \\
%     0 & 0 & 0 & 0 & \gamma_m
% \end{bmatrix}
% \begin{bmatrix}
%     \bm{u}_1^\intercal \\
%     \bm{u}_2^\intercal\\
%     \vdots\\
%     \bm{u}_m^\intercal\\
% \end{bmatrix}\\
% &=
% \Big[\bm{v}_1 \big|\bm{v}_2 \big| \bm{v}_3 \big| \dots \big| \bm{v}_m \Big]
% \begin{bmatrix}
% \gamma_1 \bm{u}_1^\intercal \\
% \gamma_2 \bm{u}_2^\intercal\\
%     \vdots\\
% \gamma_m \bm{u}_m^\intercal\\
% \end{bmatrix}\\
% \end{align*}

% \begin{align*}
%     \{\bm{U\Gamma V^\intercal}\}_{j,k}&= \sum_{i=1}^m \gamma_i \bm{v}_{ji}\bm{u}_{ik}\\
%     &= 
% \end{align*}



% % \begin{bmatrix}
% %     \gamma_1 & 0 & 0 & 0 & 0 & \dots  & 0 \\
% %     0 & \gamma_2 & 0 & 0 & 0 & \dots  & 0 \\
% %     \vdots & \vdots & \ddots &  &\vdots & \ddots& \vdots \\
% %     0 & 0 & 0 & \gamma_m & 0 &\dots  & 0\\
% %     \vdots & \vdots & \vdots & \vdots & \vdots & \ddots& \vdots \\
% %     0 & 0 & 0 & 0 & 0 & \dots  & 0\\
% % \end{bmatrix}
\end{document}
