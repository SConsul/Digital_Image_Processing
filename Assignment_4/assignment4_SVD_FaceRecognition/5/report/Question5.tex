\documentclass{article}
\usepackage[utf8]{inputenc}
\usepackage{bm}
\usepackage{relsize}
\usepackage{amsmath,amssymb,amsthm}
\usepackage{geometry}
\title{CS 663 Assignment 4}
\author{Sarthak Consul - 16D100012\\
Parthasarathi Khirwadkar - 16D070001\\
Bhishma Dedhia - 16D170005}
\date{}
\geometry{
    a4paper,
    total={170mm,257mm},
    left=12mm, 
    right=12mm, 
    top=20mm,
    }
\begin{document}

\maketitle

\centerline{\textbf{\large\large{{{QUESTION 5}}}}}

\begin{align*}
\centering
    J(\bm{e}) &= \sum_{i=1}^N ||\bm{x_i} -\bm{\bar{x}}-\bm{e}^\intercal (\bm{x_i} -\bm{\bar{x}})\bm{e}||^2\\
    &= \sum_{i=1}^N ||\bm{x_i} -\bm{\bar{x}}||^2 - \bm{e}^\intercal\bm{S}\bm{e} \hspace{10pt}(\bm{S} = (N-1)\bm{C})
\end{align*}
Thus the direction, $\bm{e}$,  for which $J(\bm{e})$ is minimized is that which maximizes $\bm{e}^\intercal\bm{C}\bm{e}$.\\
Let the eigenvalues of S be $\lambda_1 > \lambda_1 > \dots > \lambda_{n-1}$\\
To find the $\bm{f}$ that maximizes $J(\bm{f})$ given $\bm{f} \perp \bm{e}$ and $||\bm{f}|| = 1$, applying the Lagrangian Multiplier Test,\\
\[
\mathlarger{\mathlarger{\nabla}}_{f}\Big[ \bm{f}^\intercal\bm{Sf} - \lambda (\bm{f}^\intercal\bm{f} - 1) -\beta (\bm{e}^\intercal\bm{f}) \Big] = 0 \]
\[ 2\bm{S}\bm{f} - 2\lambda\bm{f} - \beta\bm{e} = 0\]
Pre-multipying both sides with $\bm{e}^\intercal$,
\[ 2(\bm{S}^\intercal\bm{e})^\intercal\bm{f} - 2\lambda\bm{e}^\intercal\bm{f} - \beta\bm{e}^\intercal\bm{e} = 0\]
As $\bm{f} \perp \bm{e}$, $\bm{e}^\intercal\bm{f} = 0$. Also $\bm{S}$ is symmetric and so $\bm{S}^\intercal\bm{e} = \bm{S}\bm{e} = \lambda_1\bm{e}$ and finally  $\bm{e}^\intercal\bm{e} = 1$
\[\implies 2\lambda_1\bm{e}^\intercal\bm{f} - \beta = 0\]
\[\implies \beta = 0\]
\[\text{Thus }\bm{Sf} = \lambda \bm{f} \hspace{3pt} \big(\text{i.e. }\bm{f} \text{ is an eigenvector of }\bm{S} \big)\]
Let the eigenvalue of $\bm{S}$ corresponding to $\bm{f}$ be $\mu$
\[J(\bm{f}) = \mu^2 \hspace{5pt}(\text{As }\bm{f}\text{ is of unit norm})\]
Maximizing $J(\bm{f})$ would mean that $\bm(f)$ corresponds to the largest eigenvalue, given that it should be orthogonal to $\bm{e}$.\\
$\mu \neq \lambda_1$ as it is given that $\bm{C}$ and so $\bm{S}$ has distinct eigenvalues and if $\mu = \lambda_1$, that would mean $\bm{f} = \bm{e}$ (and so $\bm{f} \not\perp \bm{e}$) $\Rightarrow\mskip-\thinmuskip\Leftarrow$\\
The next best choice of $\mu$ is $\lambda_2$ and since eigenvectors of symmetric matrices are orthogonal, no contradictions arise.\\
Thus $\bm{f}$ has to be the eignevector corresponding to the second largest eigenvalue 

\end{document}
